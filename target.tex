\documentclass{article}

\usepackage[utf8]{inputenc}
\usepackage[margin=1in]{geometry}   % Set margins.
\usepackage{graphicx}               % Allows adding images to the document.
\usepackage{float}                  % A positioning package for images, charts, and tables.
\usepackage{listings}               % Useful for importing code snippets into reports.
\usepackage{xcolor}                 % Colorize some text.
\usepackage{caption}                % Caption directives.
\usepackage{courier}                % Courier font.
\usepackage{fancyhdr}               % Header.
\usepackage{titling}                % Author info access.
\usepackage{multirow}               % Multirow in tables.
\usepackage{hyperref}               % Hyperlinks.
\usepackage[bottom]{footmisc}       % Footnote at bottom.
\usepackage{longtable}              % Multipage tables.
% No indent, space after paragraphs.
\setlength{\parindent}{0pt}
\setlength{\parskip}{1em}
\renewcommand\arraystretch{1.5}

% Header setup.
\pagestyle{fancy}
\fancyhf{}
\rhead{\thetitle}
\lhead{Pg. \thepage}

% URL setup.
\definecolor{cerulean}{rgb}{0.0, 0.48, 0.65}
\hypersetup{
    colorlinks=true,
    linkcolor=blue,
    filecolor=magenta,
    urlcolor=cerulean,
    citecolor=blue
}

% Title centered.
%\renewcommand\maketitlehooka{\null\mbox{}\vspace{5cm}}
%\renewcommand\maketitlehookd{\vfill\null}

% Title left aligned.
\makeatletter
\renewcommand{\maketitle}{
    \bgroup\setlength{\parindent}{0pt}
    \begin{flushleft}
    \huge\textbf{\@title}
    \vspace*{0.2cm}

    \large\@author\\
    \@date
    \end{flushleft}\egroup
}
\makeatother

% New command to generate URL links and footers.
\newcommand\fnurl[2]{
    \href{#2}{#1}\footnote{\url{#2}}
}

% Fake item for lists in tables.
\newcommand{\tabitem}{\textbullet \hspace{0.5em}}

% Language setup.
\lstset{
    language=C++,
    morekeywords={fmt, ornl, util, std, vector, string},
    tabsize=4,
    numbers=left,
    firstnumber=1,
    xleftmargin=2em,
    frame=single,
    basicstyle=\footnotesize\ttfamily,
    keywordstyle=\color{teal!50!black}\bf,
    commentstyle=\color{gray},
    numberstyle=\color{gray}\bf,
    stringstyle=\color{purple}\bf,
    breaklines=true,
    postbreak=\mbox{\textcolor{red}{$\hookrightarrow$}\space},
    backgroundcolor=\color{gray!5},
}


% Author information
\title{The Title}
\author{The Author}
\date{\today}

\begin{document}
    \maketitle
    \thispagestyle{empty}

    This is some example text to show how to use the template.

    For writing mathematical equations, it is much easier to use:
    \[
    	\alpha\beta = \frac{dx}{dt}\gamma\Omega
    \]

    Figures can be added using the figure environment:
    \begin{figure}[H]
        \centering
        \includegraphics[width=0.5\columnwidth]{example-image-a}
        \caption{Example caption for this figure.}
    \end{figure}

    Example code listing:
    \begin{lstlisting}[gobble=8]
        #pragma once

        // C++
        #include <vector>
        #include <string>

        // libfmt
        #include <fmt/core.h>

        namespace ornl::util {
            template<
                typename T,
                typename return_t = std::vector<typename std::vector<T>::pointer>
            >
            return_t make_pointer_container(std::vector<T>& vector) {
                return_t result;
                for (T& element : vector) {
                    result.push_back(&element);
                }

                return result;
            }

            void test_function() {
                std::vector<std::string> strings = {"one", "two", "three"};
                auto pointers = make_pointer_container(strings);

                for (const auto& pointer : pointers) {
                    fmt::print("{}: {}\n", pointer, *pointer);
                }

                fmt::print("This is a really long line that shows how the wrapping works using mbox hookarrow. This text should be on the next line, if it isn't that's bad.\n");
            }

        }
    \end{lstlisting}

    You can also include code from a file using \verb|\lstinputlisting{filename.ext}|.

\end{document}
