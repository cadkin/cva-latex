\documentclass{article}

\include{preamble.tex}

% Author information
\title{The Title}
\author{The Author}
\date{\today}

\begin{document}
    \maketitle
    \thispagestyle{empty}

    This is some example text to show how to use the template.

    For writing mathematical equations, it is much easier to use:
    \[
    	\alpha\beta = \frac{dx}{dt}\gamma\Omega
    \]

    Figures can be added using the figure environment:
    \begin{figure}[H]
        \centering
        \includegraphics[width=0.5\columnwidth]{example-image-a}
        \caption{Example caption for this figure.}
    \end{figure}

    Example code listing:
    \begin{lstlisting}[gobble=8]
        #pragma once

        // C++
        #include <vector>
        #include <string>

        // libfmt
        #include <fmt/core.h>

        namespace ornl::util {
            template<
                typename T,
                typename return_t = std::vector<typename std::vector<T>::pointer>
            >
            return_t make_pointer_container(std::vector<T>& vector) {
                return_t result;
                for (T& element : vector) {
                    result.push_back(&element);
                }

                return result;
            }

            void test_function() {
                std::vector<std::string> strings = {"one", "two", "three"};
                auto pointers = make_pointer_container(strings);

                for (const auto& pointer : pointers) {
                    fmt::print("{}: {}\n", pointer, *pointer);
                }

                fmt::print("This is a really long line that shows how the wrapping works using mbox hookarrow. This text should be on the next line, if it isn't that's bad.\n");
            }

        }
    \end{lstlisting}

    You can also include code from a file using \verb|\lstinputlisting{filename.ext}|.

\end{document}
